\documentclass{article}
%% PACKAGES %%

\usepackage{amsmath, amsfonts, amssymb, amsthm}
\usepackage{braket}
\usepackage{listings}
\usepackage{geometry}
\usepackage{xcolor}
\usepackage{textcomp}
\usepackage{graphicx}
\usepackage{fancyhdr}
\usepackage{sourcecodepro}
\usepackage{multirow}

%%%%%%%%%%%%%%

\graphicspath{{./images}}

%% LISTINGS CONFIG %%

\definecolor{purple2}{RGB}{153,0,153} % there's actually no standard purple
\definecolor{green2}{RGB}{0,153,0} % a darker green

\lstset{
  language=Verilog,                   % the language
  basicstyle=\normalsize\ttfamily,   % size of the fonts for the code
  frame = single,
  % Color settings to match IDLE style
  keywordstyle=\color{orange},       % core keywords
  keywordstyle={[2]\color{purple2}}, % built-ins
  stringstyle=\color{green2},%
  showstringspaces=false,
  commentstyle=\color{red},%
  upquote=true,                      % requires textcomp
  numbers=left,
  breaklines=true,
}

% Title Stuff
\title{\vspace{-3cm} \(\beta\)-Gallium Oxide Device Simulations in Synopsys \\ End of Spring 2024 Documentation}
\author{Chase A. Lotito, \textit{SIUC Undergraduate}}
\date{}

\begin{document}

\pagestyle{fancy}

% attempt to make nice header
\fancyhead{}
\fancyhead[CH]{\normalsize{LOTITO - SIUC NANO ECE - SPRING 2024}}

\maketitle % Makes the title

\section{Introduction}

The following outlines the workflow I have developed using \textit{Synopsys Sentarus} for device simulation using \(\beta - \text{Ga}_2 \text{O}_3\) as the semiconductor material. 

This outlines adding a new material to the complete \textit{models.par} file, ensuring the new material is a defined variable in the \textit{datexcodes.txt} file, what parameters were changed inside the specific \(\beta - \text{Ga}_2 \text{O}_3\) parameter (.par) file, and how I proceeded to simulate the specific devices in breakdown.

\section{Creating a custom material in Sentarus}

There are two tools from Synopsys that we are actively using to construct and simulate a semiconductor device. To contstruct the device, we use the Synopsys Structure Editor (bash cmd: sde). To simulate the device, we use Synopsys Device (bash cmd: sdevice). 

\subsection{datexcodes.txt}

In order for Synopsys Structure Editor to know what materials you can construct a device from, it checks the defined variables in the \textit{datexcodes.txt} file, which is located (/synopsys/TCAD/tcad/J-2014.09/lib). You can create another \textit{datexcodes.txt} file locally in your current project's directory, and Synopsys will prioritize the variables defined in that local file. 

What I did was copy the \textit{datexcodes.txt} file from (/synopsys/TCAD/tcad/J-2014.09/lib), and append the following code:

\begin{lstlisting}
Materials {
    BetaGalliumOxide {
    label = "BetaGalliumOxide"
    group = Semiconductor
    color = #c8a2c8, #e1ceff
     }
    
    ! the rest of the file 
}
\end{lstlisting}


\end{document}
