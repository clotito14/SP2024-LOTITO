\documentclass{IEEEtran}

% Packages
\usepackage{amsmath}
\usepackage[cmintegrals]{newtxmath}
\usepackage{graphicx}
\usepackage{xurl} % Makes urls better
\usepackage{listings}

% MATLAB typesetting
\usepackage{color} %red, green, blue, yellow, cyan, magenta, black, white
\definecolor{mygreen}{RGB}{28,172,0} % color values Red, Green, Blue
\definecolor{mylilas}{RGB}{170,55,241}

% Title Stuff
\title{Designing a 350nm Ultraviolet Light Emitter with GaN Nanocrystal}
\author{Chase A. Lotito, \textit{SIUC Undergraduate}}
\date{}

% Makes a header!
\markboth{ECE447 --- Semiconductor Devices --- Project 1, February 2024}{Shell \MakeLowercase{\text
it{et al.}}: A Novel Tin Can Link}

\begin{document}

% Config for using listings
\lstset{language=Matlab,%
    %basicstyle=\color{red},
    basicstyle = \ttfamily,
    frame = single,
    breaklines=true,%
    morekeywords={matlab2tikz},
    keywordstyle=\color{blue},%
    morekeywords=[2]{1}, keywordstyle=[2]{\color{black}},
    identifierstyle=\color{black},%
    stringstyle=\color{mylilas},
    commentstyle=\color{mygreen},%
    showstringspaces=false,%without this there will be a symbol in the places where there is a space
    emph=[1]{for,end,break},emphstyle=[1]\color{red}, %some words to emphasise
    %emph=[2]{word1,word2}, emphstyle=[2]{style},    
}

\maketitle % Makes the title

% ABSTRACT
\begin{abstract}
    % [A brief statement on what you plan to do in this project.]

    Using an online simulator, we wish to make a \(350nm\) ultraviolet light emitting device out of \(GaN\) nanocrystal. By changing the height of the box nanocrystal, we will quantumly constrain the electrons and cause their bandgap energy to change to the energy needed for a 350nm-wavelength photon.   

\end{abstract}

\section{Introduction}
% [Why this project and related device is important.]

From Einstein, we know photons of discrete frequencies are emitted from de-energizing electrons. If we could harness these emitted photons, maybe we could get wavelength-specific light sources that are more efficient than their incandescent counterparts.

Ultraviolet light-sources have uses in sanitation, medical, and law enforcement. Manufacturing with a material like $GaN$ will be a large step forward to improving our qualities of life. With efficient use of resources, these nanocrystal devices can find uses in developing nations who can be provided cheaper nanoscale electronic devices.

So, we need to find a way to engineer the specific wavelength of light we need. Our goal is to manipulate the quantum confinement on the electrons in a \(GaN\) box-shaped nanocrystal.

Ultraviolet radiation's wavelength falls in between \(400nm\) and \(1nm\). To get the design frequency \(f_d = 350nm\), we would need higher frequency ultraviolet light from our \(GaN\) nanocrystal. We can find the exact frequency for light in free-space:

\begin{equation}
    f = \frac{c}{\lambda}
\end{equation}

\begin{equation*}
    \implies f_d = \frac{3 \times 10^8}{350 \times 10^{-9}} = 8.571 \times 10^{14} Hz
\end{equation*}

From here, we can use the Planck-Einstein relation, to find the necessary bandgap energy \(E_G\).

\begin{equation}
    E_G = hf
\end{equation}

\begin{equation*}
    \implies E_G = (6.626 \times 10^{-34})(8.571 \times 10^{14}) = 3.545 eV
\end{equation*}

Here, \(E_G\) really represents the difference in energy from the first-excited energy state \(E_1\) and the ground energy state \(E_0\), \(h = 6.626 \times 10^{-34} ~ J\cdot s\), and f is the frequency of the particle.

Now, it is not certain that nature allows for a continuous frequency \(GaN\) electrons. For a given crystal, we know from Schrodinger's wave equation \(\Psi\) that the energy states for its electrons has been discretized by the potential boundaries it imposes. So, if we can change the potentials we can tailor \(E_G\) to our desired frequency.

Ultimately, we achieve this by manipulating the volume of a cube \(GaN\) nanocrystal by changing it's height \(L_z\). The different heights might constrain or relieve the wavefunction of the electrons giving rise to different bandgaps.


\section{Design Method}
% [i.e. the simulator used and a reference to it.]

From nanoHUB, the \textit{Nanoscale Solid-State Lighting Device Simulator} \cite{simulator} from Southern Illinois University Carbondale, we can simulate GaN nanocrystals of different shapes, calculating the 3-D wavefunctions for the device.

We will sweep the height of the nanocrystal \(L_z\) from 1nm-6nm, and use Microsoft EXCEL to find a polynomial regression \(E_G(L_z)\) that approximates the bandgap energy \(E_G\) with respect to \(L_z\).

The resulting function for bandgap will let us predict the value for \(L_z\) for \(E_G = 3.545eV\). It will also be valuable to see how \(dE_G / dL_z\) will show intuition on quantum constraining.

\section{Discussion on Design and Analysis}
% [Describe design and analysis. attach necessary graphs, snapshots of simulator pages, and/or codes if used.]

From our sweep across values of \(L_z\), we were able to get the following polynomial approximation for \(E_G\) with respect to \(L_z\) (\(R^2 = 0.9946\)):

\begin{multline} \label{bandgap}
    E_G(L_z) = 0.0002 L_z^6 - 0.0077 L_z^5 + 0.112 L_z^4 \\ - 0.8339 L_z^3 + 3.3261 L_z^2 - 6.7689 L_z + 9
\end{multline}

The following MATLAB code allows us to plot our function and eventually its derivative:

\begin{lstlisting}
function [] = BandgapCrystalHeightPlot()
    syms x
    t = 0:0.1:6;
    f = @(x) 0.0002*x.^6 - 0.0077*x.^5 + 0.112*x.^4 - 0.8339*x.^3 + 3.3261*x.^2 - 6.7689*x + 9;
    df= matlabFunction(diff(f(x),1));

    %plot(t, f(t), 'B.--'); hold on;
  
    plot(t, f(t), 'R.--'); hold on;
    ylim([0,6]);
    plot(t, 3.545, 'K.--');

    xlabel('Nanocrystal Height (nm)');
    ylabel('Bandgap Energy per Unit Height (eV/nm)');
   % ylabel('dEg/dz (eV/nm)');
 
    grid("on");
    title('First Derivative of Bandgap Energy by Nanocrystal Height');
end
\end{lstlisting}

% Graph of bandgap function
\begin{figure}[!h] 
    \centering
    \includegraphics*[width = 6cm]{EgVLz.png}
    \caption{\(E_G\) versus \(L_z\)}
    \label{fig:func}
\end{figure}    

Using MATLAB, we can find when \(L_z = 2.3nm\), \(E_G(2.3) = 3.545eV\).

Now, the simulator only calculates the wavefunction for integer multiples of \(0.5nm\), so we cannot verify our value for \(L_z\), but a close approximation is fine. 

For this reason, the height we will use in the simulator will be \(L_z = 2.5nm\), and this gets us a bandgap \(E_G = 3.552eV\) (this means \(\lambda = 349.33nm \approx 350nm\)).



% Talking about the derivative

\subsection{First Derivative of \(E_G\)}

We can also use MATLAB to graph the derivative of Eq. \ref{bandgap}. 

What we find is the bandgap is most sensitive to changes in the nanocrystal height for values of \(L_z\) between \(0nm - 2nm\).

This is the region that the quantum confinement is stronger than the natural confinement that a bulk piece of \(GaN\) provides to an electron. 

We would need a better approximation for Eq. \ref{bandgap}, but we can predict that as \(L_z \to \infty\), we would find \(E_G \to E_{G0}\), where \(E_{G0}\) is the \textit{bulk bandgap energy} of \(GaN\) .



% Graph of the derivative
\begin{figure}[!h] 
    \centering
    \includegraphics*[width = 6cm]{derivativeOfEgVLz.png}
    \caption{\(dE_G / dL_z\)}
    \label{fig:derivative}
\end{figure}    

\subsection{Why does the bandgap vary with \( L_z \)?}

Remember for the wavefunction solved for a electron in an infinite potential well, the wavenumber \(k_n\) was:

\begin{equation}
    k_n = \frac{n \pi}{L}
\end{equation}

This equation showing that the wavenumber is inversely proportional to the dimensions of the quantum confinement. And, in the k-space, energyis inversely proportional to that dimension squared.

\begin{equation}
    E \propto 1/L^2
\end{equation}

This proportionality follows with what we have in Fig. \ref{fig:func}. Differences in the actual bandgap energy function we simulated comes from the specific crystal lattice with crystal potential \(U_c\).

Speaking in terms of the Heisenberg Uncertainty principle, the narrower the confinement of our electrons, the more certain their position can be in all of space, and this causes a larger spread for the electron's momentum. As \(\Delta p \propto \Delta E\), we get more energy for more localized particles. Then, by providing different dimensional constraints on the electron, we can engineer the correct bandgap energy for 350nm light.

\subsection{Absorption}

Figure \ref{fig:absorption} shows the changes in light absorption when sweeping the light source from \(0^\circ\) to \(360^\circ\). The curve that shows the most absorption is at \(\phi \approx 92.5^\circ\).

Absorption describes the which wavelengths of light are actually transmitted when incident on a device or material. For nanoscale optoelectronics, the devices should absorb the most light at energies equal to the bandgap or above.

Figure \ref{fig:absorption} shows absorption across all light energy levels, therefore all frequencies, with peaks at 3.552 eV, 3.583 eV, and 3.672 eV. 

The differences and broadening of the absorption curve can be explained by the asymmetry or anisotropy of the device, and by Pauli's Exclusion Principle.

The anisotropy of \(GaN\) means light will interact differently with the device when incident from different angles since device characteristics will measure differently based on direction. 

Also, Pauli's Exclusion Principle dictates that no two electrons can have the same quantum numbers. So, when our \(GaN\) atoms are closely-packed in our nanoscale crystal, the multitude of electrons enter hybridized orbitals and are separated into slightly different energy levels. A splitting of energy levels occurs. This causes the spread of absorption levels.

% Absorption with phi parameteric sweep
\begin{figure}[!h] 
    \centering
    \includegraphics*[width = 6cm]{AbsorptionSweepAnglephi.png}
    \caption{Absorption with \(\phi\) paramteric sweep.}
    \label{fig:absorption}
\end{figure}    

\subsection{3-D Wavefunctions}

\begin{figure}[!h] 
    \centering
    \includegraphics*[width = 6cm]{energyone.jpg}
    \caption{Ground State Wavefunction.}
    \label{fig:energyone}
\end{figure}

Using the \textit{3-D Wavefunctions} tab, we can see the different orbitals that take shape for our nanocrystal. Figure \ref{fig:energyone} showsthe ground-state wavefunction, which takes the shape of an \(s\)-orbital. Similarly for the first-excited state wavefunction, we also see a \(s\)-orbital in Fig. \ref{fig:energytwo}.

\begin{figure}[!h] 
    \centering
    \includegraphics*[width = 6cm]{energytwo.jpg}
    \caption{First-Excited Energy Wavefunction.}
    \label{fig:energytwo}
\end{figure}    

As we can see, the difference between the ground-state and first-excited energy state wavefunctions is their size. Since the first-excited state has more energy, the electron has a better chance of being farther away from the potential well that it is captured in. This highlights how the \(GaN\) nanocrystal is acting like an artificial atom, which can trap electrons and have them distributed in orbitals much like a naturally occuring atom given the correct potentials.

The third energy state is when we see our first \(p\)-orbital, which takes the form of a dumbbell shape. 

\subsection{Integrated Absorption}

Using the \textit{Integrated Absorption} tab, we can see how the device's absorption behaves as a function of \(\phi\). And, as we can see in Fig. \ref{fig:integratedabsorption}, the absorption is maximized for \(\phi_1 = 92.5^\circ\) and \(\phi_2 = 267.4^\circ\). The graph itself resembling a fully-rectified sinusoid.  

\begin{figure}[!t] 
    \centering
    \includegraphics*[width = 6cm]{IntegratedAbsorption2.jpg}
    \caption{Absorption with respect to \(\phi\).}
    \label{fig:integratedabsorption}
\end{figure} 

But observing the \(s\)-orbitals for the ground-state and the first-excited energy states, and eventually the \(p\)-orbital, setting the origin to the center of the wavefunctions, the ellipsoid shapes cause the probability of finding the electron highest at \(\phi_n \approx (2n+1)(90^\circ)\). Since the electron is most likely at these poles, absorption has to be maximized here as well; since at any other location a photon has less of a chance of interacting with an electron. 

\section{Conclusion}
% [State what you have learned].

In conclusion, to create a nanocrystal that emits 350nm light, we can create a \(GaN\) box-shaped nanocrystal with dimensions of \(10nm \times 10nm \times 2.5nm\). This configuration gives a bandgap \(E_G = 3.552eV\), and this lands us very close to our wavelength goal. 

This experiment is vital in showing how quantum confinment can be used to engineer the bandgap \(E_G\) of a semiconductor material. \(GaN\)  is normally used for blue LED technology, but the use of different semiconductors and or manipulating the geometry can produce optoelectronic devices for any optical needs. 

Also, nanoscale devices like these are great ways to see how to make artificial atoms. The 3-D solutions to the Schrodinger equation show strong parallels to the wavefunctions that result from naturally occuring atoms. And since the wavefunctions in our nanocrystal were not spherically symmetric, we were able to easily explain the irregularity of absorption around the azimuth of the device.

% REFERENCES!
\bibliographystyle{IEEEtran}
\bibliography{project1Bib.bib}

\end{document}
